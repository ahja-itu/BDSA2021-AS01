% https://da.overleaf.com/latex/templates/cs310-assignment-template/qrqpndrxpcht
%%%%%%%%%%%%%%%%% DO NOT CHANGE HERE %%%%%%%%%%%%%%%%%%%% {
\documentclass[12pt,letterpaper]{article}
\usepackage{fullpage}
\usepackage[top=2cm, bottom=4.5cm, left=2.5cm, right=2.5cm]{geometry}
\usepackage{amsmath,amsthm,amsfonts,amssymb,amscd}
\usepackage{lastpage}
\usepackage{enumerate}
\usepackage{fancyhdr}
\usepackage{mathrsfs}
\usepackage{xcolor}
\usepackage{graphicx}
\usepackage{listings}
\usepackage{hyperref}
\usepackage{listings}
\usepackage{float}
\usepackage{wrapfig}

\definecolor{BackgroundColor}{rgb}{0.9,0.9,0.9}
\definecolor{OliveGreen}{rgb}{0,0.6,0}

\lstset{
  basicstyle=\normalsize\fontencoding{T1}\ttfamily,
  language=C++,
  backgroundcolor=\color{BackgroundColor},
  tabsize=4,
  captionpos=b,
  %tabsize=3,
  frame=lines,
  numbers=left,
  numberstyle=\tiny,
  numbersep=5pt,
  breaklines=true,
  showstringspaces=false,
  keywordstyle=\color{blue},
  commentstyle=\color{OliveGreen},
  stringstyle=\color{red}
  }

\hypersetup{%
    colorlinks=true,
    linkcolor=blue,
    linkbordercolor={0 0 1}
}

\renewcommand{\labelenumii}{\theenumii}
\renewcommand{\theenumii}{\theenumi.\arabic{enumii}.}

\setlength{\parindent}{0.0in}
\setlength{\parskip}{0.05in}
%%%%%%%%%%%%%%%%%%%%%%%%%%%%%%%%%%%%%%%%%%%%%%%%%%%%%%%%%% }

%%%%%%%%%%%%%%%%%%%%%%%% CHANGE HERE %%%%%%%%%%%%%%%%%%%% {
\newcommand\course{BDSA2021}
\newcommand\semester{\today}
\newcommand\hwnumber{01}         % <-- ASSIGNMENT #
\newcommand\NetIDa{Andreas Wachs Hjalager}      % <-- YOUR NAME
\newcommand\NetIDb{19167}      % <-- STUDENT ID #
%%%%%%%%%%%%%%%%%%%%%%%%%%%%%%%%%%%%%%%%%%%%%%%%%%%%%%%%%% }

%%%%%%%%%%%%%%%%% DO NOT CHANGE HERE %%%%%%%%%%%%%%%%%%%% {
\pagestyle{fancyplain}
\headheight 35pt
\lhead{\NetIDa}
\lhead{\NetIDa\\Student ID: \NetIDb}
\chead{\textbf{\Large Assignment \hwnumber}}
\rhead{\course \\ \semester}
\lfoot{}
\cfoot{}
\rfoot{\small\thepage}
\headsep 1.5em
%%%%%%%%%%%%%%%%%%%%%%%%%%%%%%%%%%%%%%%%%%%%%%%%%%%%%%%%%% }

\lstdefinestyle{sharpc}{language=[Sharp]C}
\lstset{style=sharpc}

\begin{document}
\section{C\#}
\subsection{Generics}
The type constraints for the first version of \lstinline{GreaterCount}, it constraints the type
\lstinline{T} to be implementing the \lstinline{IComparable} interface, ensuring that items of type \lstinline{T}
can be compared between themselves. The generic type \lstinline{U} is not having any constraints applied to it,
but it is neither used in this case. 

Additionally, the \lstinline{item} object type needs to be implementing \lstinline{IEnumerable} with the 
generic type \lstinline{T} provided.

The type constraints for the second version of \lstinline{GreaterCount}, we achieve the same result as the above answer,
but through transitivity where the constraints are that \lstinline{T} is of same type of \lstinline{U} and type \lstinline{U} 
implements the \lstinline{IComparable} type with type \lstinline{U}. This is only accepted if the \lstinline{IComparable} interface
allows for implementations for the type \lstinline{U}.

\section{Software Engineering}
\subsection{Exercise 1}


\subsection{Exercise 2}


\subsection{Exercise 3}


\subsection{Exercise 4}


\subsection{Exercise 5}


\subsection{Exercise 6}


\end{document}

    